\chapter{Why should we introduce computing in science courses}
\section{Introduction: Scientific and educational motivation}

Many important recent advances in our understanding of the physical
world have been driven by large-scale computational modeling and data
analysis, for example, the 2012 discovery of the Higgs boson, the 2013
Nobel Prize in chemistry for computational modeling of molecules, and
the 2016 discovery of gravitational waves.  Given the ubiquitous use
in science and its critical importance to the future of science and
engineering, scientific computing  plays a central role in scientific
investigations and is critical to innovation in most domains of our
lives. It underpins the majority of today's technological, economic
and societal feats. We have entered an era in which huge amounts of
data offer enormous opportunities. \href{{http://pathways.acm.org/executive-summary.html}}{By 2020, it is also expected that one out of every two
jobs in the STEM (Science, Technology, Engineering and Mathematics) fields will be in
computing}
(Association for Computing Machinery, 2013, \cite{ACM2013}).

These developments, needs and future challenges, as well as the
developments that are now taking place within quantum computing,
quantum information theory and data driven discoveries (data analysis and
machine learning) will play an essential role in shaping future
technological developments. Most of these developments require true
cross-disciplinary approaches and bridge a vast range of
temporal and spatial scales and include a wide variety of physical
processes. To develop computational tools for such complex
systems that give physically meaningful insights requires a deep
understanding of approximation theory, high performance computing, and
domain specific knowledge of the area one is modeling.

Computing competence represents a central element in scientific
problem solving, from basic education and research to essentially
almost all advanced problems in modern societies. These
competencies are not limited to STEM fields only. The statistical
analysis of big data sets and how to use machine learning algorithms
belong to the set of tools needed by almost all disciplines,
spanning from the Social Sciences, Law, Education to the traditional
STEM fields and Life Science.  Unfortunately, many of our students at
both the undergraduate and the graduate levels are unprepared to use
computational modeling, data science, and high performance computing,
skills that are much valued by a broad range of employers. This lack of preparation is most certainly no fault of our students, but rather a broader issue associated with how departments, colleges, and universities are keeping up with the demands of these high-tech employers. It is through this integrated computational perspective that we aim to address this.
Furthermore, although many universities do offer compulsory
programming courses in scientific computing, and physics departments
offer one or more elective courses in computational physics, there is
often not a uniform and coherent approach to the development of
computing  competencies and computational thinking. This has
consequences for a systematic introduction and realization of
computing  skills and competencies and pertaining learning outcomes.

The aim of this contribution is to present examples on how to
introduce a computational perspective in basic undergraduate physics
courses, basing ourselves on experiences made at the University of
Oslo in Norway and now also at Michigan State University in the
USA. In particular, we will present the \textbf{Computing in Science
Education} project from the University of Oslo \cite{CSEUiO}, a project which has
evolved into a Center of Excellence in Education, the \href{{http://www.mn.uio.no/ccse/english/}}{Center for
Computing in Science
Education} \cite{CCSEUiO}. Similar initiatives
and ideas are also being pursued at Michigan State University.  The
overarching aim is to strengthen the computing  competencies of
students, with key activities such as the establishment of learning
outcomes, how to develop assessment programs and course
transformations by including computational projects and exercises in a
coherent way. The hope is that these initiatives can also lead to a
better understanding of the scientific method and scientific reasoning
as well as providing new and deeper insights about the underlying physics that governs a system.

This contribution is organized as follows. After these introductory
remarks, we present briefly in the next section what we mean by computing and present
possible learning outcomes that could be applied to a bachelor's degree
program in physics (Sec. \ref{sec:competencies}), which are distinguished as more general competencies and course-specific ones.  In Sec. \ref{sec:learingoutcomes}, we discuss possible paths on how to include and implement
computational elements in central undergraduate physics courses. We discuss briefly
how to assess various learning outcomes and how to develop a research program around this.
Several examples that illustrate the links between the learning outcomes and specific mathematics and physics courses
are discussed in Sec. \ref{sec:examples}.
Finally, in the last section we present our conclusions and perspectives.


\section{Computing competencies} \label{sec:competencies}

The focus of this article is on computing competencies and how
these help in enlarging the body of tools available to students and
scientists alike, going well beyond classical tools taught in standard
undergraduate courses in physics and mathematics. We will claim through various
examples that computing  allows for a more generic handling of
problems, where focusing on algorithmic aspects results in deeper
insights about scientific problems.

With \textbf{Computing } we will mean solving scientific problems
using all possible tools, including symbolic computing, computers and
numerical algorithms, experiments (often of a numerical character) and
analytical paper and pencil solutions. We will thus, deliberately,
avoid a discussion of computing and computational physics in
particular as something separate from theoretical physics and
experimental physics.  It is common in the scientific literature to
encounter statements like \emph{Computational physics now represents
  the third leg of research alongside analytical theory and
  experiments}. In selected contexts where say high-performance topics
or specific computational methodologies play a central role, it may be
meaningful to separate analytical work from computational studies. We
will however argue strongly, in particular within an educational
context, for a view where computing means solving scientific problems
with all possible tools. Through various examples in this article we
will show that a tight connection between standard analytical work,
combined with various algorithms and a computational approach, can
help in enhancing the students' understanding of the scientific
method, hopefully providing deeper insights about the physics (or
other disciplines). Whether and how we achieve these outcomes is the
purpose of research in computational physics education.

The power of the scientific method lies in identifying a given problem
as a special case of an abstract class of problems, identifying
general solution methods for this class of problems, and applying a
general method to the specific problem (applying means, in the case of
computing, calculations by pen and paper, symbolic computing, or
numerical computing  by ready-made and/or self-written software).

This generic view on problems and methods is particularly important for
understanding how to apply available generic software to solve a
particular problem.  Algorithms involving pen and paper are
traditionally aimed at what we often refer to as continuous models, of
which only few can be solved analytically. The number of important
differential equations in physics that can be solved analytically are
rather few, limiting thereby the set of problems that can be addressed
in order to deepen a student's insights about a particular physics
case. On the other hand, the application of computers calls for
approximate discrete models.  Much of the development of methods for
continuous models are now being replaced by methods for discrete
models in science and industry, simply because we can address much
larger classes of problems with discrete models, often also by simpler
and more generic methodologies.  In Sec. \ref{sec:examples} we will present
several examples thereof. A typical case is that where an eigenvalue
problem can allow students to study the analytical solution as well as
moving to an interacting quantum mechanical case where no analytical
solution exists. By merely changing the diagonal matrix elements, one
can solve problems that span from classical mechanics and fluid
dynamics to quantum mechanics and statistical physics.  Using
essentially the same algorithm one can study physics cases that
are covered by several courses, allowing teachers to focus
more on the physical systems of interest.

There are several advantages in  introducing computing in basic physics
courses. It allows physics teachers to bring important elements of
scientific methods at a much earlier stage in our students'
education. Many advanced simulations used in physics research can
easily be introduced, via various simplifications, in introductory
physics courses, enhancing thereby the set of problems studied by the
students (see Sec. \ref{sec:examples}). Computing gives university
teachers a unique opportunity to enhance students' insights about
physics and how to solve scientific problems. It gives the
students the skills and abilities that are asked for by
society. Computing allows for solving more realistic problems earlier
and can provide an excellent training of creativity as well as enhancing the
understanding of abstractions and generalizations. Furthermore,
computing can decrease the need for special tricks and tedious
algebra, and shifts the focus to problem definition, visualization,
and "what if" discussions. Finally, if the setup of undergraduate
courses is properly designed, with a synchronization with mathematics
and computational science courses, computing can trigger further
insights in mathematics and other disciplines.

