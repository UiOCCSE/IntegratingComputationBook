\chapter{From Mechanics to Electromagnetism}

A simple rewrite allows for the reuse in linear algebra problems for
solution of say Poisson's equation in electromagnetism, or the
diffusion equation in one dimension. To see this and how the same matrix can be used in a course in electromagnetism, let us consider
Poisson's equation.
We assume that
the electrostatic potential $\Phi$ is generated by a localized charge
distribution $\rho (\mathbf{r})$.   In three dimensions
the pertinent equation reads
\[
\nabla^2 \Phi = -4\pi \rho (\mathbf{r}).
\]
With a spherically symmetric potential $\Phi$ and charge distribution $\rho (\mathbf{r})$ and using spherical coordinates,  the relevant
equation to solve
simplifies to a one-dimensional equation in $r$, namely
\[
\frac{1}{r^2}\frac{d}{dr}\left(r^2\frac{d\Phi}{dr}\right) = -4\pi \rho(r),
\]
which can be rewritten via a substitution $\Phi(r)= \phi(r)/r$ as
\[
\frac{d^2\phi}{dr^2}= -4\pi r\rho(r).
\]
The inhomogeneous term $f$ or source term is given by the charge distribution
$\rho$  multiplied by $r$ and the constant $-4\pi$.

We can  rewrite this equation by letting $\phi\rightarrow u$ and
$r\rightarrow x$.  Scaling again the equations and replacing the right hand side with a function $f(x)$, we can rewrite the
equation as
\[
-u''(x) = f(x).
\]
Our scaling gives us again $x\in [0,1]$ and the two-point boundary value problem
with $u(0)=u(1)=0$. With $n+1$ integration points and
the step length defined as $h=1/(n)$ and replacing the continuous function $u$ with its discretized version $v$, we get
the following equation
\begin{equation*}
   -\frac{v_{i+1}+v_{i-1}-2v_i}{h^2} = f_i  \hspace{0.5cm} \mathrm{for} \hspace{0.1cm} i=1,\dots, n,
\end{equation*}
where $f_i=f(x_i)$.
Bringing up again the tridiagonal Toeplitz matrix,
\[
    \mathbf{A} = \frac{1}{h^2}\begin{bmatrix}
                           2& -1& 0 &\dots   & \dots &0 \\
                           -1 & 2 & -1 &0 &\dots &\dots \\
                           0&-1 &2 & -1 & 0 & \dots \\
                           & \dots   & \dots &\dots   &\dots & \dots \\
                           0&\dots   &  &-1 &2& -1 \\
                           0&\dots    &  & 0  &-1 & 2 \\
                      \end{bmatrix},
\]
our problem becomes now a classical linear algebra problem
\[
\mathbf{A}\mathbf{v}=\mathbf{f},
\]
with the unknown function $\mathbf{v}$. Using standard LU
decomposition algorithms \cite{GolubVanLoan} (here one can use
the so-called Thomas algorithm which reduces the number of floating
point operations to $O(n)$) one can easily find the solution to this
problem.

These examples demonstrate how one can, with a discretized second
derivative, solve physics problems that arise in different
undergraduate courses using standard linear algebra and eigenvalue
algorithms and ordinary differential equations, allowing thereby
teachers to focus on the interesting physics. Many of these problems
can easily be linked up with ongoing research. This opens up for many
interesting perspectives in physics education. We can bring in at a
much earlier stage in our education basic research elements and
perhaps even link with ongoing research during the first year of
undergraduate studies.

Instead of focusing on tricks and mathematical manipulations to solve
the continuous problems for those few case where an analytical
solution can be found, the discretization of the continuous problem
opens up for studies of many more interesting and realistic problems.
However, we have seen that in order to verify and validate our codes,
the existence of analytical solutions offer us an invaluable test of
our algorithms and programs. The analytical results can either be
included explicitely or via symbolic software like Python's Sympy package.
Thus, computing stands indeed for solving scientific problems using
all possible tools, including symbolic computing, computers and
numerical algorithms, numerical experiments (as well as real
experiments if possible) and analytical paper and pencil solutions.

