\chapter{Developing an Assessment Program}

\section{Physics Education Research and Computing in Science Education}
The introduction of computational elements in the various courses should be, if possible,  strongly integrated with ongoing research on physics education.
The Physics and Astronomy department at MSU is in a unique position due to its strong research group in physics education, the \href{{http://www.pa.msu.edu/research/physics-education-lab}}{PERL group} \cite{PERLMSU}. Together with the Center for Computing in Science Education at the University of Oslo \cite{CCSEUiO}, we are now in the process
of establishing new assessments
and assessment methods that address several issues associated with
integrating computing into science courses. The issues include but
are not limited to how well students learn computing, what new
insights students gain about the specific science through computing,
and how students' affective states (e.g., motivation to learn,
computational self-efficacy) are affected by computing . Broadly
speaking, these assessments should provide deeper insights into the
integration of computing  in science education in general as well as
provide a structured framework for assessment of our efforts and a
basis for systematic studies of student learning.

The central questions that our research must address are
\begin{enumerate}
\item how can we assess the effect of integrating computing  into science curricula on a variety of learned-centered constructs including computational thinking, motivation, self-efficacy and science identity formation,
\item how should we structure assessments to ensure valid, reliable and impactful assessment, which provides useful information to our program and central partners, and finally
\item how can the use of these structured assessments improve student outcomes in teacher-, peer-, and self-assessment.
\end{enumerate}


Addressing these questions requires a combination of qualitative
techniques to construct the focus of these assessments, to build
assessment items and to develop appropriate assessment methods, and
quantitative techniques, including advanced statistical analysis to
ensure validity and reliability of the proposed methods as well as to
analyze the resulting data.

The learning objectives and learning outcomes for computational
methods developed as part of the first objective form parts of the
basis for the assessment program, and we will also investigate the
assessment of non-content learning goals such as self-efficacy and
identity formation. Identifying and investigating the role of such non-content
factors will be critical to support all students in achieving our computational
learning goals.

The effect of integration of computational methods into basic science
courses have been sparsely studied, primarily because the practice is
sparse. Further progress depends now on the development of assessments
that can be used for investigative, comparative and/or longitudinal
studies and to establish best practices in this emerging field.  Some
assessments will be developed for specific courses, but we will aim
for broad applicability across institutions.

