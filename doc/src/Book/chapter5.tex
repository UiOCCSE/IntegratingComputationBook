\chapter{Teaching Mechanics with Computational Exercises and Projects}


We assume that our students know how to solve and study systems of
ordinary differential with initial conditions only. Later in this
section we will venture into two-point boundary value problems that
can be studied and solved with eigenvalue solvers.

Let us start with initial value problems and ordinary differential
equations. Such equations appear in a wealth of physics
applications. Typical examples students will encounter are the
classical pendulum in a mechanics course, an RLC circuit in the course on
electromagnetism, the modeling of the Solar system in an Astrophysics
course and many other cases.  The essential message is that, with
properly scaled equations, students can use essentially the
same algorithms to solve these problems, either starting with
a simple modified Euler algorithm or a Runge-Kutta class of
algorithms or the so-called Verlet class of algorithms, to mention a few.

The idea is that algorithms students develop and use in one course can be
reused in other courses.  This allows  students to make the
relevant abstractions discussed above, opening up for a much wider
range of applicabilities.

Here we look at two familiar cases from
mechanics and electromagnetism, the equations for the classical
pendulum and those for an RLC circuit.  When properly scaled, these
equations are essentially the same. To scale equations,
either in terms of dimensionless variables or appropriate variables,
is an important aspect which allows the students to see the potential
for abstractions and hopefully see how the problems studied in say a
mechanics course can be transferred to other fields.

The classical pendulum with damping and external force as it could
appear in a mechanics course is given by the following equation of
motion for the angle $\theta$ as function of time $t$
\[
  ml\frac{d^2\theta}{dt^2}+\nu\frac{d\theta}{dt}  +mgsin(\theta)=Acos(\omega t),
\]
where $m$ is its mass, $l$ the length, $\nu$ a damping factor and $A$
the amplitude of an applied external source with frequency
$\omega$. The solution of this type of equations (second-order
differential equations with given initial conditions) is something the
students encounter the first semester thorugh the courses IN1900 and
MAT-INF1100 at the University of Oslo. At Michigan State University
there is now a compulsory course for physics majors that includes many
of these elements.  With this background, students are already familiar with
the numerical solution and visualization of such equations.
If we now
move to a course on electromagnetism, we encounter almost the same
equation for an RLC circuit, namely
\[
L\frac{d^2Q}{dt^2}+\frac{Q}{C}+R\frac{dQ}{dt}=Acos(\omega t),
\]
where $L$ is the inductance, $R$ the applied resistance, $Q$ the
time-dependent charge and $C$ the capacitance.

Let us consider first the classical pendulum equations with damping and an
external force and define the scaled velocity $\hat{v}$ as
\[
   \frac{d\theta}{d\hat{t}} =\hat{v},
\]
where we have defined a dimensionless time variable $\hat{t}$. With
the equation for the velocity we can rewrite the second-order
differential in terms of two coupled first-order differential
equations where the second equation represents the acceleration
\[
   \frac{d\hat{v}}{d\hat{t}} =Acos(\hat{\omega} \hat{t})-\hat{v}\xi-\sin(\theta).
\]
We have scaled the  equations with $\omega_0=\sqrt{g/l}$,
$\hat{t}=\omega_0 t$ and $\xi = mg/\omega_0\nu$. The frequency
$\omega_0$ defines a so-called natural frequency defined by the
gravitational acceleration $g$ and the length of the pendulum $l$. The
frequency $\hat{\omega}= \omega/\omega_0$.  In a similar way, our RLC
circuit can now be rewritten in terms of two coupled first-order
differential equations,
\[
   \frac{dQ}{d\hat{t}} =\hat{I},
\]
and
\[
   \frac{d\hat{I}}{d\hat{t}} =Acos(\hat{\omega} \hat{t})-\hat{I}\xi-Q,
\]
with $\omega_0=1/\sqrt{LC}$, $\hat{t}=\omega_0 t$ and $\xi =
CR\omega_0$. Here we see that the natural frequency is defined in
terms of the physical parameters $L$ and $C$.

The equations are essentially the same, the main differences reside in
the different scaling constants and the introduction of a non-linear
term for the angle $\theta$ in the pendulum equation. The differential
solver the students end up writing in the mechanics course (which
comes normally before the course on electromagnetism) can then be
reused in the electromagnetism course, with a great potential for
further abstraction.

Let us now move to another frequently encountered problem in several
physics courses, namely that of a two-point boundary value problem. In
the examples below we will see again that if the equations are
properly scaled, we can reuse the same algorithm for solving different
physics problems. Here we will start with the equations for a buckling
beam (a case which can be found in a mechanics course or a course on
mathematical methods in physics). Thereafter, with a simple change of
variables and constants, the same problem can be used to study a
quantum mechanical particle confined to move in an infinite potential
well.  By simply changing the diagonal matrix elements of the
discretized differential equation problem, we can study particles that
move in a harmonic oscillator potential or other types of
quantum-mechanical one-body or selected two-body problems.  With
slight modifications to the matrix that results from the
discretization of a second derivative, we can study Poisson's equation
in one dimension, a problem of relevance in
electromagnetism.

Let us start with the buckling beam. This is a two-point boundary
value problem
\[
R \frac{d^2 u(x)}{dx^2} = -F u(x),
\]
where $u(x)$ is the vertical displacement, $R$ is a material specific
constant, $F$ is the applied force and $x \in [0,L]$ with $u(0)=u(L)=0$.
We scale the equation with $x = \rho L$ and $\rho \in [0,1]$ and get
(note that we change from $u(x)$ to $v(\rho)$)
\[
\frac{d^2 v(\rho)}{dx^2} +K v(\rho)=0,
\]
which is, when discretized (see below), nothing but a standard eigenvalue problem with $K=
FL^2/R$. Here we can assume that either the force $F$ or the material
specific rigidity $R$ are unknown.  If we replace $R=-\hbar^2/2m$ and
$-F=\lambda$, we have the quantum mechanical variant for a particle
moving in a well with infinite walls at the endpoints.  The way to
solve these equations numerically is to discretize the second
derivative and the right hand side as
\[
    -\frac{v_{i+1} -2v_i +v_{i-i}}{h^2}=\lambda v_i,
\]
with $i=1,2,\dots, n$. Here $h$ is the step size which is defined by
the number of integration (or mesh) points.  We need to add to this
system the two boundary conditions $v(0) =v_0$ and $v(1) = v_{n+1}$,
although they are not needed in the solution of the equations since
their values are known.  For all integration points $i=1,2,\dots, n$
the set of equations to solve result in a so-called tridiagonal
Toeplitz matrix ( a special case from the discretized second
derivative)
\[
    \mathbf{A} = \frac{1}{h^2}\begin{bmatrix}
                          2 & -1 &  &   &  & \\
                          -1 & 2 & -1 & & & \\
                           & -1 & 2 & -1 & &  \\
                           & \dots   & \dots &\dots   &\dots & \dots \\
                           &   &  &-1  &2& -1 \\
                           &    &  &   &-1 & 2 \\
                      \end{bmatrix}
\]
and with the corresponding vectors $\mathbf{v} = (v_1, v_2,
\dots,v_n)^T$ allows us to rewrite the differential equation as a
standard eigenvalue problem
\[
   \mathbf{A}\mathbf{v} = \lambda\mathbf{v}.
\]
The tridiagonal Toeplitz matrix has analytical eigenpairs,
providing us thereby with an invaluable check on the equations to be
solved.

