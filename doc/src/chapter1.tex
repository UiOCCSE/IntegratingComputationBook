\chapter{What do we mean by computing}
\section*{Introduction: Scientific and educational motivation}

Numerical simulations of various systems in science are central to our
basic understanding of nature and technology.
The increase in computational power,
improved algorithms for solving problems in science, as well as access
to high-performance facilities, allow researchers to study
complicated systems across many length and energy scales. Applications
span from studying quantum physical systems in nanotechnology and the
characteristics of new materials or subatomic physics at its smallest
length scale, to simulating galaxies and the evolution of the universe.
In between, simulations are key to understanding
cancer treatment and how the brain works,
predicting climate changes and this week's weather,
simulating natural disasters, semi-conductor devices,
quantum computers, as well as assessing risk in the insurance and
financial industry.


\paragraph{Computing competence.}
Computing means solving scientific problems using computers. It covers
numerical as well as symbolic computing. Computing is also about
developing an understanding of the scientific process by enhancing
algorithmic thinking when solving problems.  Computing competence has
always been a central part of education in the sciences and engineering disciplines.

On the part of students, this competence involves being able to:

\begin{itemize}
\item understand how algorithms are used to solve mathematical problems,

\item derive, verify and implement algorithms,

\item understand what can go wrong with algorithms,

\item use these algorithms to construct reproducible scientific outcomes and to engage in science in ethical ways, and

\item think algorithmically for the purposes of gaining deeper insights about scientific problems.
\end{itemize}

\noindent
All these elements are central for maturing and gaining a better understanding of the scientific process.

The power of the scientific method lies in identifying a given problem
as a special case of an abstract class of problems, identifying
general solution methods for this class of problems, and applying a
general method to the specific problem (applying means, in the case of
computing, calculations by pen and paper, symbolic computing, or
numerical computing by ready-made and/or self-written software). This
generic view on problems and methods is particularly important for
understanding how to apply available, generic software to solve a
particular problem.

Computing competence represents a central element
in scientific problem solving, from basic education and research to
essentially almost all advanced problems in modern
societies. Computing competence is central to further
progress. It enlarges the body of tools available to students and
scientists beyond classical tools and allows for a more generic
handling of problems. Focusing on algorithmic aspects results in
deeper insights about scientific problems.





\paragraph{Why should basic university education undergo a shift towards modern computing?}
\begin{itemize}
\item Algorithms involving pen and paper are traditionally aimed at what we often refer to as continuous models.

\item Application of computers calls for approximate discrete models.

\item Much of the development of methods for continuous models are now being replaced by methods  for discrete models in science and industry, simply because much larger classes of problems can be addressed with discrete models, often also by simpler and more generic methodologies.
\end{itemize}

\noindent
However, verification of algorithms and understanding their limitations requires much of the classical knowledge about continuous models.

So, why should basic university education undergo a shift towards modern computing?

The impact of the computer on mathematics and science is tremendous: science and industry now rely on solving mathematical problems through computing.
\begin{itemize}
\item Computing can increase the relevance in education by solving more realistic problems earlier.

\item Computing through programming can be excellent training of creativity.

\item Computing can enhance the understanding of abstractions and generalization.

\item Computing can decrease the need for special tricks and tedious algebra, and shifts the focus to problem definition, visualization, and "what if" discussions.
\end{itemize}

\noindent
The result is a deeper understanding of mathematical modeling and the scientific method (we hope, and here our physics education research group can play a central role in promoting this).
Not only is computing via programming a very powerful tool, it can also be a great pedagogical aid.

For the mathematical training, there is one major new component among the arguments above: understanding abstractions and generalization. While many of the classical methods developed for continuous models are specialized for a particular problem or a narrow class of problems, computing-based algorithms are often developed for problems in a generic form and hence applicable to a large problem class.


Computing competence represents a central element in scientific problem solving, from basic education and research to essentially almost all advanced problems in modern societies. It enlarges the body of tools available to students and scientists beyond classical tools and allows for a more generic handling of problems. Focusing on algorithmic aspects results in deeper insights about scientific problems.

Moreover, today's projects in science and industry tend to involve larger teams. Tools for reliable collaboration must therefore be mastered (e.g., version control systems, automated computer experiments for reproducibility, software and method documentation).


